\documentclass[10pt,twocolumn,oneside]{article}
\setlength{\columnsep}{18pt}                    %兩欄模式的間距
\setlength{\columnseprule}{0pt}                 %兩欄模式間格線粗細

\usepackage{amsthm}                             %定義,例題
\usepackage{amssymb}
\usepackage{fontspec}                           %設定字體
\usepackage{color}
\usepackage[x11names]{xcolor}
\usepackage{listings}                           %顯示code用的
\usepackage{fancyhdr}                           %設定頁首頁尾
\usepackage{graphicx}                           %Graphic
\usepackage{enumerate}
\usepackage{titlesec}
\usepackage{amsmath}
\usepackage[CheckSingle, CJKmath]{xeCJK}
\usepackage{CJKulem}

\usepackage{amsmath, courier, listings, fancyhdr, graphicx}
\topmargin=0pt
\headsep=5pt
\textheight=740pt
\footskip=0pt
\voffset=-50pt
\textwidth=545pt
\marginparsep=0pt
\marginparwidth=0pt
\marginparpush=0pt
\oddsidemargin=0pt
\evensidemargin=0pt
\hoffset=-42pt

%\renewcommand\listfigurename{圖目錄}
%\renewcommand\listtablename{表目錄}

%%%%%%%%%%%%%%%%%%%%%%%%%%%%%

\setmainfont[
    AutoFakeSlant,
    BoldItalicFeatures={FakeSlant},
    UprightFont={* Medium},
    BoldFont={* Bold}
]{Inconsolata}
%\setmonofont{Ubuntu Mono}
\setmonofont[
    AutoFakeSlant,
    BoldItalicFeatures={FakeSlant},
    UprightFont={* Medium},
    BoldFont={* Bold}
]{Inconsolata}
\setCJKmainfont{Noto Sans CJK TC}
\XeTeXlinebreaklocale "zh"                      %中文自動換行
\XeTeXlinebreakskip = 0pt plus 1pt              %設定段落之間的距離
\setcounter{secnumdepth}{3}                     %目錄顯示第三層

%%%%%%%%%%%%%%%%%%%%%%%%%%%%%
\makeatletter
\lst@CCPutMacro\lst@ProcessOther {"2D}{\lst@ttfamily{-{}}{-{}}}
\@empty\z@\@empty
\makeatother
\lstset{                                        % Code顯示
    language=C++,                               % the language of the code
    basicstyle=\footnotesize\ttfamily,          % the size of the fonts that are used for the code
    numbers=left,                               % where to put the line-numbers
    numberstyle=\scriptsize,                    % the size of the fonts that are used for the line-numbers
    stepnumber=1,                               % the step between two line-numbers. If it's 1, each line  will be numbered
    numbersep=5pt,                              % how far the line-numbers are from the code
    backgroundcolor=\color{white},              % choose the background color. You must add \usepackage{color}
    showspaces=false,                           % show spaces adding particular underscores
    showstringspaces=false,                     % underline spaces within strings
    showtabs=false,                             % show tabs within strings adding particular underscores
    frame=false,                                % adds a frame around the code
    tabsize=2,                                  % sets default tabsize to 2 spaces
    captionpos=b,                               % sets the caption-position to bottom
    breaklines=true,                            % sets automatic line breaking
    breakatwhitespace=true,                     % sets if automatic breaks should only happen at whitespace
    escapeinside={\%*}{*)},                     % if you want to add a comment within your code
    morekeywords={*},                           % if you want to add more keywords to the set
    keywordstyle=\bfseries\color{Blue1},
    commentstyle=\itshape\color{Red1},
    stringstyle=\itshape\color{Green4},
}


\begin{document}
\pagestyle{fancy}
\fancyfoot{}
%\fancyfoot[R]{\includegraphics[width=20pt]{ironwood.jpg}}
\fancyhead[C]{FJCU}
\fancyhead[L]{H2J}
\fancyhead[R]{\thepage}
\renewcommand{\headrulewidth}{0.4pt}
\renewcommand{\contentsname}{Contents}

\scriptsize
\tableofcontents
\section{Basic}
    \subsection{Run}
        \lstinputlisting{Contents/Basic/Run.sh}
    \subsection{Ternary Search}
        \lstinputlisting{Contents/Basic/Ternary_Search.cpp} 

\section{Data Structure}
    \subsection{BIT RARSQ}
        \lstinputlisting{Contents/Data_Structure/BIT_RARSQ.cpp}
    \subsection{zkw RMQ}
        \lstinputlisting{Contents/Data_Structure/zkw_RMQ.cpp}
    \subsection{Segment Tree RARMQ}
        \lstinputlisting{Contents/Data_Structure/SegTree_RARMQ.cpp}
    \subsection{Treap}
        \lstinputlisting{Contents/Data_Structure/Treap.cpp}

\section{Graph}
    \subsection{Directed MST}
        \lstinputlisting{Contents/Graph/Directed_MST.cpp}
    \subsection{LCA}
        \lstinputlisting{Contents/Graph/LCA.cpp}
    \subsection{Euler Circuit}
        七橋問題根據起點與終點是否相同,分成 Euler path(不同)及 Euler circuit(相同)。

\begin{itemize}
    \item 判斷法
    \item 無向圖部分,將點分成奇點(度數為奇數)和偶點(度數為偶數)。
    \begin{itemize}
        \item Euler path:奇點數為 0 或 2
        \item Euler circuit:沒有奇點
    \end{itemize}
    \item 有向圖部分,將點分成出點(出度 - 入度 = 1)和入點(入度 - 出度 = 1)還有平衡點(出度 = 入度)。
    \begin{itemize}
        \item Euler path:出點和入點個數同時為 0 或 1。
        \item Euler circuit:只有平衡點。
    \end{itemize}
    \item 求出一組解
    \item 用 DFS 遍歷整張圖,設 S 為離開的順序,無向圖的答案為 S ,有向圖的答案為反向的 S 。
    \item DFS 起點選定:
    \begin{itemize}
        \item Euler path:無向圖選擇任意一個奇點,有向圖選擇出點。
        \item Euler circuit:任意一點。
    \end{itemize}
\end{itemize}

\begin{lstlisting}
// Code from Eric
#define ll long long
#define PB push_back
#define EB emplace_back
#define PII pair<int, int>
#define MP make_pair
#define all(x) x.begin(), x.end()
#define maxn 50000+5
//structure
struct Eular {
  vector<PII> adj[maxn];
  vector<bool> edges;
  vector<PII> path;
  int chk[maxn];
  int n;
  void init(int _n) {
    n = _n;
    for (int i = 0; i <= n; i++) adj[i].clear();
    edges.clear();
    path.clear();
    memset(chk, 0, sizeof(chk));
  }
  void dfs(int v) {
    for (auto i : adj[v]) {
      if (edges[i.first] == true) {
        edges[i.first] = false;
        dfs(i.second);
        path.EB(MP(i.second, v));
      }
    }
  }
  void add_Edge(int from, int to) {
    edges.PB(true);
    // for bi-directed graph
    adj[from].PB(MP(edges.size() - 1, to));
    adj[to].PB(MP(edges.size() - 1, from));
    chk[from]++;
    chk[to]++;
    // for directed graph
    // adj[from].PB(MP(edges.size()-1, to));
    // check[from]++;
  }
  bool eular_path() {
    int st = -1;
    for (int i = 1; i <= n; i++) {
      if (chk[i] % 2 == 1) {
        st = i;
        break;
      }
    }
    if (st == -1) {
      return false;
    }
    dfs(st);
    return true;
  }
  void print_path(void) {
    for (auto i : path) {
      printf("%d %d\n", i.first, i.second);
    }
  }
};
\end{lstlisting}

\begin{lstlisting}
// Code from allen(lexicographic order)
#include <bits/stdc++.h>
using namespace std;
const int ALP = 30;
const int MXN = 1005;
int n;
int din[ALP], dout[ALP];
int par[ALP];
vector<string> vs[MXN], ans;
bitset<MXN> vis, used[ALP];
void djsInit() {
  for (int i = 0; i != ALP; ++i) {
    par[i] = i;
  }
}
int Find(int x) { return (x == par[x] ? (x) : (par[x] = Find(par[x]))); }
void init() {
  djsInit();
  memset(din, 0, sizeof(din));
  memset(dout, 0, sizeof(dout));
  vis.reset();
  for (int i = 0; i != ALP; ++i) {
    vs[i].clear();
    used[i].reset();
  }
  return;
}
void dfs(int u) {
  for (int i = 0; i != (int)vs[u].size(); ++i) {
    if (used[u][i]) {
      continue;
    }
    used[u][i] = 1;
    string s = vs[u][i];
    int v = s[s.size() - 1] - 'a';
    dfs(v);
    ans.push_back(s);
  }
}
bool solve() {
  int cnt = 1;
  for (int i = 0; i != n; ++i) {
    string s;
    cin >> s;
    int from = s[0] - 'a', to = s.back() - 'a';
    ++din[to];
    ++dout[from];
    vs[from].push_back(s);
    vis[from] = vis[to] = true;
    if ((from = Find(from)) != (to = Find(to))) {
      par[from] = to;
      ++cnt;
    }
  }
  if ((int)vis.count() != cnt) {
    return false;
  }
  int root, st, pin = 0, pout = 0;
  for (int i = ALP - 1; i >= 0; --i) {
    sort(vs[i].begin(), vs[i].end());
    if (vs[i].size()) root = i;
    int d = dout[i] - din[i];
    if (d == 1) {
      ++pout;
      st = i;
    } else if (d == -1) {
      ++pin;
    } else if (d != 0) {
      return false;
    }
  }
  if (pin != pout || pin > 1) {
    return false;
  }
  ans.clear();
  dfs((pin ? st : root));
  return true;
}
int main() {
  int t;
  cin >> t;
  while (t--) {
    cin >> n;
    init();
    if (!solve()) {
      cout << "***\n";
      continue;
    }
    for (int i = ans.size() - 1; i >= 0; --i) {
      cout << ans[i] << ".\n"[i == 0];
    }
  }
}
\end{lstlisting}

\section{Connectivity}
    \subsection{Articulation Point}
        \lstinputlisting{Contents/Connectivity/Art_Point.cpp}
    \subsection{Bridges}
        \lstinputlisting{Contents/Connectivity/Bridges.cpp}

\section{Flow \& Matching}
    \subsection{Relation}
        \lstinputlisting{Contents/Flow_and_Matching/Relation.txt}
    \subsection{Bipartite Matching}
        \lstinputlisting{Contents/Flow_and_Matching/Bipartite_Matching.cpp}
    \subsection{KM}
        \lstinputlisting{Contents/Flow_and_Matching/KM.cpp}
    \subsection{Dinic}
        \lstinputlisting{Contents/Flow_and_Matching/Dinic.cpp}
    \subsection{MCMF}
        \lstinputlisting{Contents/Flow_and_Matching/MCMF.cpp}

\section{String}
    \subsection{Manacher}
        \lstinputlisting{Contents/String/Manacher.cpp}
    \subsection{Trie}
        \lstinputlisting{Contents/String/Trie.cpp}

\section{Math}
    \subsection{Number Theory}
        \begin{itemize}
    \item Inversion:\\ $aa^{-1} \equiv 1 \pmod{m}$. $a^{-1}$ exists iff $\gcd(a,m)=1$.
    \item Linear inversion:\\ $a^{-1} \equiv (m - \lfloor\frac{m}{a}\rfloor) \times (m \bmod a)^{-1} \pmod{m}$
    \item Fermat's little theorem:\\ $a^p \equiv a \pmod{p}$ if $p$ is prime.
    \item Euler function:\\ $\phi(n)=n \prod_{p|n} \frac{p-1}{p}$
    \item Euler theorem:\\ $a^{\phi(n)} \equiv 1 \pmod{n}$ if $\gcd(a,n) = 1$.
    \item Extended Euclidean algorithm:\\
    $ax+by=\gcd(a,b)=\gcd(b, a \bmod b)=\gcd(b, a-\lfloor\frac{a}{b}\rfloor b)=bx_1+(a-\lfloor\frac{a}{b}\rfloor b)y_1=ay_1+b(x_1-\lfloor\frac{a}{b}\rfloor y_1)$
    \item Divisor function:\\ $\sigma_x(n) = \sum_{d|n}d^x$. $n=\prod_{i=1}^r p_i^{a_i}$.\\ $\sigma_x(n)=\prod_{i=1}^r \frac{p_i^{(a_i+1)x}-1}{p_i^x-1}$ if $x \neq 0$. $\sigma_0(n)=\prod_{i=1}^r (a_i+1)$.
    \item Chinese remainder theorem:\\ $x \equiv a_i \pmod{m_i}$.\\
        $M=\prod m_i$. $M_i=M/m_i$. $t_i=M_i^{-1}$.\\
        $x = kM + \sum a_i t_i M_i$, $k \in \mathbb{Z}$.
\end{itemize}
    \subsection{Extended GCD}
        \lstinputlisting{Contents/Math/Extended_GCD.cpp}
    \subsection{Gaussian Elimination}
        \lstinputlisting{Contents/Math/Gaussian_Elimination.cpp}
    \subsection{Phi}
        \begin{itemize}
    \item 歐拉函數計算對於一個整數 N,小於等於 N 的正整數中,有幾個和 N 互質
    \item 如果 $gcd(p, q) = 1, \Phi(p) \cdot \Phi(q) = \Phi(p \cdot q)$
    \item $\Phi(p^k) = p^{k - 1} \times (p - 1)$
\end{itemize}

\begin{lstlisting}
void phi_table(int n) {
  phi[1] = 1;
  for (int i = 2; i <= n; i++) {
    if (phi[i]) {
      continue;
    }
    for (int j = i; j < n; j += i) {
      if (!phi[j]) {
        phi[j] = j;
      }
      phi[j] = phi[j] / i * (i - 1);
    }
  }
}
\end{lstlisting}

\section{Geometry}
    \subsection{Point}
        \lstinputlisting{Contents/Geometry/Point.cpp}
    \subsection{Line}
        \[d(P, L) = \frac{|ax_0 + by_0 + c|}{\sqrt[]{a^2 + b^2}}\]

\begin{lstlisting}
struct Line {
  Pt st;
  Pt ed;
};
// return point side
// left, on line, right -> 1, 0, -1
int side(Line l, Pt a) {
  dvt cross_val = cross(a - l.st, l.ed - l.st);
  if (cross_val > EPS) {
    return 1;
  } else if (cross_val < -EPS) {
    return -1;
  } else {
    return 0;
  }
}
// AB infinity, CD segment
bool has_intersection(Line AB, Line CD) {
  int c = side(AB, CD.st);
  int d = side(AB, CD.ed);
  if (c == 0 || d == 0) {
    return true;
  } else {
    // different side
    return c == -d;
  }
}
// find intersection point, two line, not seg
pair<int, Pt> intersection(Line a, Line b) {
  Pt A = a.ed - a.st;
  Pt B = b.ed - b.st;
  Pt C = b.st - a.st;
  dvt mom = cross(A, B);
  dvt son = cross(C, B);
  if (std::abs(mom) <= EPS) {
    if (std::abs(son) <= EPS) {
      return {1, {}}; // same line
    } else {
      return {2, {}}; // parallel
    }
  } else {            // ok
    return {0, a.st + A * (son / mom)};
  }
}
// line to point distance
dvt dis_lp(Line l, Pt a) {
  return area3x2(l.st, l.ed, a) / dis_pp(l.st, l.ed);
}
\end{lstlisting}
    \subsection{Area}
        \lstinputlisting{Contents/Geometry/Area.cpp}
    \subsection{Convex Hull}
        \lstinputlisting{Contents/Geometry/Convex_Hull.cpp}


\end{document}
